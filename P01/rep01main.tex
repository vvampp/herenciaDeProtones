\documentclass[12pt]{article}
\usepackage[a4paper, total={6in, 8in}]{geometry}
\usepackage[spanish]{babel}
\usepackage{graphicx,adjustbox,wrapfig,fancyhdr,xcolor,lipsum,parskip,multicol,amsmath,textgreek,enumitem}
\graphicspath{{./imgs/}}
\pagenumbering{arabic}
\pagestyle{fancy}
\setlength\headheight{41pt} 
\rhead{{\color{blue}\rule{1cm}{1cm}}}
\rhead{\includegraphics[width=1.2cm]{Logo_IPN.png}}

\begin{document}
    \begin{titlepage}  
        \centering
        {\bfseries\LARGE Instituto Politécnico Nacional\par}
        \vspace{0.5cm}
        {\scshape\Large Escuela Superior de Computo \par}
        \vspace{2cm}
        {\scshape\Huge Reporte: Práctica 01 \par}
        \vspace{3cm}
        {\LARGE Electrónica Analógica \par}
        \vfill
        {\Large Carpio Becerra Erick Gustavo \par}
        {\Large Espinoza Vera Fransisco \par}
        {\Large Portela Nájera Jesús Bambino \par}
        \vfill
        {\Large 19 de Abril de 2023 \par}
    \end{titlepage}
    \section*{Objetivo}
    \label{sec:objetivo}
        Al termino de la práctica, el alumno comprenderá el funcionamiento de los rectificadores simples y de 
        los rectificadores con filtro de integración; y comparará los resultados obtenidos experimentalmente con los 
        obtenidos mediante simulación y teóricamente.
    \section*{Introducción}
    \label{sec:introduccion}
        \noindent Un \textbf{diodo} es un componente electrónico de dos terminales, cuya principal característica es
        su capacidad de permitir el flujo de corriente a través de él mismo en una sola dirección.
        Es crucial en muchos circuitos electrónicos, ya que permite la conversión de corriente alterna
        (AC) a corriente directa (DC) mediante la rectificación de la onda, concepto que se abordará
        con escrutinio. \par
        El principio de funcionamiento de un diodo se basa en la interacción entre sus materiales
        semiconductores de tipo N y de tipo P. Típicamente, la constitución de este componente 
        consiste de este par de tipos de semiconductores, que al unirse se convierten en una unión PN. \par
        El material tipo P está dopado con impurezas con excedente de espacios (carga positiva),
        mientras que el material tipo N está dopado con impurezas con excedente de electrones (carga negativa).\par
        Cuando se aplica un voltaje a través del diodo, los agujeros del semiconductor tipo P y 
        los electrones en el semiconductor tipo N comienzan a distribuirse sobre la unión.
        El campo eléctrico en la unión previene que la mayoría de los portadores en el semiconductor
        atraviesen la unión, pero algunos portadores minoritarios, que son los electrones en el material
        tipo P y huecos en el material tipo N, son capaces de atravesarla y recombinarse con los 
        portadores de carga opuesta en el otro lado de la unión. Este proceso produce el agotamiento de 
        portadores en la vecindad de la unión PN; en otras palabras, no habrán portadores libres en esta zona.\par
        Si el diodo está en polarización directa, es decir, si la terminal positiva está conectada al
        material tipo P, y la terminal negativa conectada al material tipo N, el voltaje aplicado al
        diodo crea un un campo eléctrico que reduce el grosor de la zona de agotamiento, permitiendo
        así que los portadores mayoritarios fluyan en la unión y en consecuencia, permitiendo el flujo 
        de corriente a través del diodo.\par
        En contraste, si el diodo está en polarización inversa, es decir, si la terminal positiva está
        conectada al material tipo N, y la terminal negativa conectada al material tipo P, el voltaje 
        aplicado incrementa considerablemente el grosor de la zona de agotamiento, evitando de esta forma
        que los portadores minoritarios en el semiconductor atraviesen la unión y por lo tanto 
        bloqueando el flujo de la corriente eléctrica en el componente.\par
        En consecuencia, el diodo juega el papel de `válvula de un solo sentido', permitiendo el flujo
        de electrones en sentido, y bloqueándolo en el otro. El voltaje requerido para polarizar el diodo
        directamente es referido como `caída de tension directa' y típicamente se encuentra el rededor
        de 0.6 a 1 voltio para diodos de silicio, y 0.2 a 0.3 voltios para diodos de germanio.\par
        Los diodos tienen numerosa cantidad de aplicaciones en circuitos eléctricos, entre los que se
        enlistan pero no se limitan a: rectificación, regulación de voltaje, detección de señales.
        Existen diferentes tipos de diodos: los estándar, diodos Zenner, diodos Schottky y LED's, cada
        uno diseñado para aplicaciones específicas. Para efectos de esta actividad, los diodos
        estándar son aquellos relevantes.\par
        
        El objetivo de un \textbf{rectificador} es convertir corriente alterna (AC) a corriente directa (DC).
        Para este punto del curso, es bien sabido que la electricidad que se transporta en las redes
        mundiales, y que llega a las tomas domesticas e industriales está en corriente alterna, es decir
        su magnitud varía en función del tiempo, mientras que la corriente directa, utilizada en vasta
        cantidad de circuitos eléctricos, tiene una magnitud y dirección constantes.\par

        El trabajo del rectificador es permitir que la corriente fluya en una sola dirección, filtrando
        la parte negativa de la onda de la corriente alterna, permitiendo el flujo solamente de la parte
        positiva. Esto, mediante el uso de diodos.\par

        Cuando la corriente alterna se aplica a un diodo en el circuito rectificador, este bloque
        el flujo de corriente durante la mitad negativa del ciclo alternante de corriente, deshaciéndose 
        de él, mientras que permite el flujo de corriente durante la mitad positiva de dicho ciclo. Este
        procedimiento convierte la forma de onda alternante a una onda directa con un voltaje pulsante.\par

        El pulso en el voltaje de la corriente ya trasformada puede ser `suavizado' mediante el uso de
        capacitores, resultando en corriente directa más estable. Cuando se coloca un capacitor en la salida del 
        transformador este pasa a denominarse transformador con filtro. Dependiendo de la capacitancia del Capacitor
        utilizado, el efecto de `suavizado' será proporcional.\par
        Dadas las características de la onda alterna y la onda rectificada las magnitudes relevantes son:\par
        \begin{description}
            \item[$V_P$] \quad \textbf{Voltaje pico}: Es el valor máximo que alcanza el voltaje en la onda de entrada.
            \item[$V_0$] \quad  \textbf{Voltaje de salida}: Se trata el promedio de voltaje en la salida del rectificador.
            \item[$I_0$] \quad  \textbf{Corriente de salida}: Es la corriente en la salida del rectificador.
            \item[$V_{max}$] \quad  \textbf{Voltaje máximo}: Se trata del valor máximo de voltaje que alcanza la onda en la salida del rectificador.
            \item[$V_{min}$] \quad  \textbf{Voltaje mínimo}: Se trata del valor mínimo de voltaje al que llega el pulso de la onda en la salida del rectificador.
            \item[$\Delta V_0$] \quad  \textbf{Voltaje de rizo}: Es la diferencia entre el voltaje máximo y el voltaje mínimo en la salida del rectificador.
        \end{description}
        Existen 3 principales tipos de circuitos rectificadores:\par
            \textbf{Rectificador de media onda.}Solo un diodo es empleado en el proceso de rectificación. Funciona utilizando de la forma más
            básica los fundamentos de funcionamiento básicos de los diodos, enlistados con anterioridad, es
            con esta aproximación a la rectificación que la corriente directa de salida es pulsante, y solamente
            aprovecha la mitad de la onda de alterna en la entrada del circuito.\par
            Las ecuaciones relevantes en el marco de esta practica referentes a este tipo de rectificador son:\par
            \begin{multicols}{2}
                \begin{itemize}
                    \item$V_p = V_{PI} = V_{rms}\sqrt{2}$
                    \item \begin{equation*}
                        V_0=(V_P-V_D)(1-\frac{1}{2fRC})
                    \end{equation*}
                    \item $I_0 = V_0 \frac{1}{R}$
                    \item $V_{max} = V_P - V_D$
                    \item \begin{equation*}
                        V_{min}=(V_P-V_D)(1-\frac{1}{fRC})
                    \end{equation*}
                    \item \begin{equation*}
                        \Delta  V_0 = \frac{V_P-V_D}{fRC}
                    \end{equation*}
                \end{itemize}
            \end{multicols}

            \textbf{Rectificador de onda completa.}  Emplea dos diodos para convertir tanto la parte positiva como la negativa de la onda de corriente alterna
            en la entrada del circuito. Esto es posible gracias a un transformador con derivación central, 
            el cual divide al voltaje alterno en dos partes iguales. El par de diodos son utilizados para 
            conducir la corriente durante cada medio ciclo de la onda AC, resultando así en una salida de 
            corriente directa más estable en comparación con la alternativa anterior.\par
            Las ecuaciones relevantes en el marco de esta practica referentes a este tipo de rectificador son:\par
            \begin{multicols}{2}
                \begin{itemize}
                    \item$V_p = V_{PI} = V_{rms}\sqrt{2}$
                    \item \begin{equation*}
                        V_0=(\frac{V_P}{2}-V_D)(1-\frac{1}{4fRC})
                    \end{equation*}
                    \item $I_0 = V_0 \frac{1}{R}$
                    \item $V_{max} = \frac{V_P}{2} - V_D$
                    \item \begin{equation*}
                        V_{min}=(\frac{V_P}{2}-V_D)(1-\frac{1}{fRC})
                    \end{equation*}
                    \item \begin{equation*}
                        \Delta  V_0 = \frac{\frac{V_P}{2}-V_D}{2fRC}
                    \end{equation*}
                \end{itemize}
            \end{multicols}


            \textbf{Rectificador de onda completa tipo puente.}Se trata de otro tipo de rectificador de onda completa, este utiliza 4 diodos en un arreglo 
            puente  para convertir la señal alterna a directa. Elimina la necesidad de un transformador
             con derivación central, beneficiando su factor de forma y relación efectividad - costo. El
             rectificador tipo puente conduce la corriente en ambas mitades del ciclo alterno, resultando así
             en la forma más estable de señal de salida directa de entre este trio de alternativas.\par
             Las ecuaciones relevantes en el marco de esta practica referentes a este tipo de rectificador son:\par
             \begin{multicols}{2}
                \begin{itemize}
                    \item$V_p = V_{PI} = V_{rms}\sqrt{2}$
                    \item \begin{equation*}
                        V_0=(V_P-2V_D)(1-\frac{1}{4fRC})
                    \end{equation*}
                    \item $I_0 = V_0 \frac{1}{R}$
                    \item $V_{max} = V_P - 2V_D$
                    \item \begin{equation*}
                        V_{min}=(V_P-2V_D)(1-\frac{1}{2fRC})
                    \end{equation*}
                    \item \begin{equation*}
                        \Delta  V_0 = \frac{V_P-2V_D}{2fRC}
                    \end{equation*}
                \end{itemize}
            \end{multicols}

        A continuación los diagramas de estos rectificadores.\newpage %eliminar después de agregar los diagramas, deben quedar en la hoja, no más%
        
        \section*{Materiales}
        \label{sec:materiales}
        \begin{itemize}
            \item [] 8 \qquad Diodos 1N4003
            \item [] 3 \qquad Resistencias de 100\textOmega\ a 10W
            \item [] 1 \qquad Transformador con derivación central de 24V/1A
            \item [] 1 \qquad Capacitor electrolítico de 470\textmu F a 50 V
            \item [] 1 \qquad Capacitor electrolítico de 2200\textmu F a 50 V
            \item [] 1 \qquad Clavija
            \item [] 1 \qquad Metro de cable 14
            \item [] 1 \qquad Protoboard
        \end{itemize}
        
        
        \section*{Desarrollo y resultados experimentales}
        \label{sec:desarrollo}
        \begin{enumerate}
            \item\textbf{Rectificador de media onda.}
            Armar el circuito de la Fig. 1.1, el transformador es alimentado de la red eléctrica de 127 Vrms a 60
            Hz.\par
            \begin{center}
                \includegraphics*[]{fig1-1.png}
            \end{center}
            Mediciones\par
            \begin{enumerate}
                \item Medir el voltaje a la salida del transformador (V1) con un multímetro en la opción CA en
                los nodos 1 y 0 del circuito, posteriormente el voltaje de la resistencia de carga (V0) en la
                opción CD del multímetro en los nodos 2 y 0, finalmente medir la corriente de salida (I0)
                en la resistencia R1 con el multímetro en la opción CD, registrar los valores obtenidos en la
                Tabla 1.1.\par
                \begin{center}
                    \includegraphics*[scale=0.55]{tabla1-1.png}
                \end{center}
                \item Mediante el osciloscopio medir el voltaje a la salida del transformador (V1) colocando el
                canal 1 del osciloscopio en los nodos 1 y 0 y medir el voltaje de salida (V0) colocando el
                canal 2 en los nodos 2 y 0. Graficar las señales que se obtienen a la entrada y la salida del
                rectificador en la Fig. 1.2, ambos canales deben de estar en el modo de CD. De la señal del
                canal 1 medir el voltaje pico (VP) a la entrada del rectificador y de la señal del canal 2 medir
                el voltaje pico menos el voltaje del diodo (VP – VD) y el voltaje de salida (V0) del rectificador,
                calcular el voltaje del diodo (VD) y la corriente de salida (I0) del rectificador, registrar los
                datos obtenidos en Tabla 1.2.
                \begin{center}
                    \includegraphics*[scale=0.75]{fig1-2.png}
                    \includegraphics*[scale=0.2]{img1-21.jpg}
                    \includegraphics*[scale=0.2]{img1-22.jpg}
                    \includegraphics*[scale=0.55]{tabla1-2.png}
                \end{center}
            \end{enumerate}
            \item\textbf{Rectificador de media onda con filtro.}\par
            Armar el circuito de la Fig. 1.3, el transformador es alimentado de la red eléctrica de 127 Vrms a 60
            Hz.\par
            \begin{center}
                \includegraphics*[scale=0.55]{fig1-3.png}
            \end{center}
            \par Mediciones\par
            \begin{enumerate}
                \item Medir el voltaje a la salida del transformador (V1) con un multímetro en la opción CA en
                los nodos 1 y 0 del circuito, medir el voltaje de la resistencia de carga (V0) con el multímetro
                en la opción CD, en los nodos 2 y 0 y medir la corriente de salida (I0) con el multímetro en
                la opción CD en la resistencia R1, registrar los valores obtenidos en la Tabla 1.3.
                \begin{center}
                    \includegraphics*[scale=0.55]{tabla1-3.png}
                \end{center}
                \item Mediante el osciloscopio medir el voltaje de entrada (V1) colocando el canal 1 en los nodos
                1 y 0 y medir el voltaje de salida (V0) colocando el canal 2 en los nodos 2 y 0, ambos canales
                deben de estar en el modo de CD. Graficar las señales obtenidas del voltaje de entrada y de
                la salida en la Fig. 1.4. De la señal del canal 1 medir el voltaje pico (VP) de la entrada del
                rectificador y de la señal del canal 2 medir el voltaje máximo (Vmax), el voltaje mínimo (Vmin)
                y el voltaje de salida (V0), además calcular la corriente de salida (I0), registrar los valores
                obtenidos en la Tabla 1.4.
                \begin{center}
                    \includegraphics*[scale=0.6]{fig1-4.png}
                    \includegraphics*[scale=0.2]{img1-41.jpg}
                    \includegraphics*[scale=0.2]{img1-42.jpg}
                    \includegraphics*[scale=0.45]{tabla1-4.png}
                \end{center}
                \item Medir el voltaje de rizo del rectificador (ΔV0), colocando el canal 2 del osciloscopio en los
                nodos 2 y 0, el canal 2 debe de estar en el modo de CA. Graficar la señal obtenida del voltaje
                de rizo en la Fig. 1.5 y medir el valor del voltaje de rizo (ΔV0), registrar el valor obtenido
                en la Tabla 1.4.
                \includegraphics*[scale=0.45]{fig1-5.png}
                Cambiar el capacitor C1 de 470 µF por un capacitor de 2,200 µF; realizar las mismas
                mediciones del inciso a) y registrar los valores obtenidos en la Tabla 1.3; realizas las mediciones del inciso b) y registrar la medición en la Tabla 1.4, además graficar las señales
                en la Fig. 1.6; y realizar las mediciones del inciso c) y registrar la medición en la Tabla 1.4,
                además graficar la señal en la Fig. 1.7.
                \begin{center}
                    \includegraphics*[scale=0.47]{fig1-6.png}
                    \includegraphics*[scale=0.45]{fig1-7.png}
                \end{center}
            \end{enumerate}
            \item \textbf{Rectificador de onda completa con derivación central.}
            Armar el circuito de la Fig. 1.8, el transformador es alimentado de la red eléctrica de 127 Vrms a
            60 Hz.
            \begin{center}
                \includegraphics*[scale=0.57]{fig1-8.png}
            \end{center}
            \par Mediciones \par
            \begin{enumerate}
                \item Medir el voltaje de las fuentes senoidal (V1 y V2) con un multímetro en la opción CA en los
                nodos 1 y 2 del circuito y el voltaje de la resistencia de carga (V0) con el multímetro en la
                opción CD en los nodos 3 y 0, además medir la corriente de salida (I0) en la resistencia R1
                con el multímetro en la opción CD, registrar los valores obtenidos en la Tabla 1.5.
                \begin{center}
                    \includegraphics*[scale=0.57]{tabla1-5.png}
                \end{center}
                \item Mediante el osciloscopio medir el voltaje de entrada (V1) colocando el canal 1 en los nodos
                1 y 0 y medir el voltaje de salida (V0) colocando el canal 2 en los nodos 3 y 0. Graficar las
                señales que se obtienen a la entrada y la salida del rectificador en la Fig. 1.9, ambos canales
                deben de estar en el modo de CD. De la señal del canal 1 medir el voltaje pico entre dos
                (VP/2) de la entrada del rectificador y de la señal del canal 2 medir el voltaje pico menos el
                voltaje del diodo (VP/2 – VD) y el voltaje de salida (V0) del rectificador, calcular el voltaje
                del diodo (VD) y la corriente de salida (I0) del rectificador, registrar los datos obtenidos en
                Tabla 1.6.
                \begin{center}
                    \includegraphics*[scale=0.57]{fig1-9.png}
                    \includegraphics*[scale=0.27]{img1-90.jpg}
                    \includegraphics*[scale=0.27]{img1-91.jpg}
                    \includegraphics*[scale=0.57]{tabla1-6.png}
                \end{center}
            
            \end{enumerate}
            \item \textbf{Rectificador de onda completa con derivación central con filtro.}\par
            Armar el circuito de la Fig. 1.10, el transformador es alimentado de la red eléctrica de 127 Vrms a
            60 Hz.\par\
            \begin{center}
                \includegraphics*[scale=0.57]{fig1-10.png}
            \end{center}
            Mediciones\par
            \begin{enumerate}
                \item Medir el voltaje de la fuente senoidal (V1 y V2) con un multímetro en la opción CA en los
                nodos 1 y 2 del circuito, medir el voltaje de la resistencia de carga (V0) con el multímetro
                en la opción CD, en los nodos 3 y 0 y medir la corriente de salida (I0) en la resistencia R1
                con el multímetro en la opción CD, registrar los valores obtenidos en la Tabla 1.7.
                \begin{center}
                    \includegraphics*[scale=0.55]{tabla1-7.png}
                \end{center}
                \item Mediante el osciloscopio medir el voltaje de entrada (V1) colocando el canal 1 en los nodos
                1 y 0; medir el voltaje de salida (V0) colocando el canal 2 en los nodos 3 y 0, ambos canales
                deben de estar en el modo de CD. Graficar las señales obtenidas del voltaje de entrada y de
                la salida en la Fig. 1.11. De la señal del canal 1 medir el voltaje pico (VP/2) de la entrada del
                rectificador y de la señal del canal 2 medir el voltaje máximo (Vmax), el voltaje mínimo (Vmin)
                y el voltaje de salida (V0), además calcular la corriente de salida (I0), registrar los valores
                obtenidos en la Tabla 1.8.
                \begin{center}
                    \includegraphics*[scale=0.45]{fig1-11.png}
                    \includegraphics*[scale=0.45]{tabla1-8.png}
                \end{center}
                \item Medir el voltaje de rizo del rectificador (ΔV0), colocando el canal 2 del osciloscopio en los
                nodos 3 y 0, el canal 2 debe de estar en el modo de CA. Graficar la señal obtenida del voltaje
                de rizo en la Fig. 1.12 y medir el valor del voltaje de rizo (ΔV0), registrar el valor obtenido
                en la Tabla 1.8.
                \begin{center}
                    \includegraphics*[scale=0.45]{fig1-12.png}
                \end{center}
                \item Cambiar el capacitor de 470 µF por un capacitor de 2,200 µF y realizar las mismas
                mediciones del inciso a), registrar los valores obtenidos en la Tabla 1.7; realizas las
                mediciones del inciso b), registrar la medición en la Tabla 1.8 y graficar las señales en la
                Fig. 1.13; y realizar las mediciones del inciso c), registrar la medición en la Tabla 1.8 y
                graficar las señales en la Fig. 1.14.
                \begin{center}
                    \includegraphics*[scale=0.45]{fig1-13.png}
                    \includegraphics*[scale=0.45]{fig1-14.png}
                \end{center}
            \end{enumerate}
            \item \textbf{Rectificador de onda completa tipo punte.}\par
            \begin{center}
                \includegraphics*[scale=0.7]{fig1-15.png}
            \end{center}
            Mediciones\par
            \begin{enumerate}
                \item Medir el voltaje de la fuente senoidal (V1) con un multímetro en la opción CA en los nodos
                1 y 2 del circuito y el voltaje de la resistencia de carga (V0) con el multímetro en la opción
                CD en los nodos 3 y 0, y medir la corriente de salida (I0) en la resistencia R1 con el
                multímetro en la opción CD, registrar los valores obtenidos en la Tabla 1.9.
                \begin{center}
                    \includegraphics*[scale=0.6]{tabla1-9.png}
                \end{center}
                \item Mediante el osciloscopio, medir el voltaje de entrada (V1) colocando el canal 1 en los nodos
                1 y 2, el canal debe de estar en modo CD y graficar la señal que se obtiene a la entrada del
                rectificador en la Fig. 1.16. De la señal del canal 1 medir el voltaje pico (VP) de la entrada
                del rectificador, registrar el valor obtenido en la Tabla 1.10.
                \begin{center}
                    \includegraphics*[scale=0.45]{fig1-16.png}
                    \includegraphics*[scale=0.45]{tabla1-10.png}
                \end{center}
                \item Desconectar el canal 1 del osciloscopio y medir el voltaje de salida (Vo) colocar el canal 2 en
                los nodos 3 y 0, el canal debe de estar en modo CD. Graficar la señal que se obtiene a la
                salida del rectificador en la Fig. 1.17. De la señal del canal 2 medir el voltaje pico menos dos
                veces el voltaje del diodo (VP – 2VD) y el voltaje de salida (V0) del rectificador, calcular el
                voltaje del diodo (VD) y la corriente de salida (I0) del rectificador, registrar los datos
                obtenidos en Tabla 1.10.
                \begin{center}
                    \includegraphics*[scale=0.5]{fig1-17.png}
                \end{center}
            \end{enumerate}
            \item \textbf{Rectificador de onda completa tipo punte con filtro.}\par
            Armar el circuito de la Fig. 1.18, el transformador es alimentado de la red eléctrica de 127 Vrms a
            60 Hz.\par
            \begin{center}
                \includegraphics*[scale=0.6]{fig1-18.png}
            \end{center}
            Mediciones\par
            \begin{enumerate}
                \item Medir el voltaje de la fuente senoidal (V1) con un multímetro en la opción CA en los nodos
                1 y 2 del circuito, medir el voltaje de la resistencia de carga (V0) con el multímetro en la
                opción CD en los nodos 3 y 0 y medir la corriente de salida (I0) en la resistencia R1 con el
                multímetro en la opción CD, registrar los valores obtenidos en la Tabla 1.11.
            
                \begin{center}
                    \includegraphics*[scale=0.6]{tabla1-11.png}
                \end{center}
                \item Mediante el osciloscopio medir el voltaje de entrada (V1) colocando el canal 1 del
                osciloscopio en los nodos 1 y 2 y graficar las señales que se obtienen a la entrada del
                rectificador de la Fig. 1.19. De la señal del canal 1 medir el voltaje pico (VP) de la entrada
                del rectificador, registrar el valor obtenido en la Tabla 1.12.
                
                \begin{center}
                    \includegraphics*[scale=0.45]{fig1-19.png}
                    \includegraphics*[scale=0.45]{tabla1-12.png}
                \end{center}
            \item Desconectar el canal 1 del osciloscopio y medir el voltaje de salida (Vo), colocando el canal
            2 en los nodos 3 y 0, graficar las señales que se obtienen a la salida del rectificador de la Fig.
            1.20, ambos canales deben de estar en el modo de CD. De la señal del canal 2 medir el
            voltaje máximo (Vmax), el voltaje mínimo (Vmin) y el voltaje de salida (V0), además calcular
            la corriente de salida (I0), registrar los valores obtenidos en la Tabla 1.12.
            \begin{center}
                \includegraphics*[scale=0.45]{fig1-20.png}
            \end{center}
            \item Medir el voltaje de rizo del rectificador (ΔV0), colocando el canal 2 del osciloscopio en los
            nodos 3 y 0, el canal 2 debe de estar en el modo de CA. Graficar la señal obtenida del voltaje
            de rizo en la Fig. 1.21, y medir el valor de voltaje de rizo (ΔV0), registrar el valor obtenido
            en la Tabla 1.12.
            \begin{center}
                \includegraphics*[scale=0.45]{fig1-21.png}
                \includegraphics*[scale=0.2]{img1-212.jpg}
            \end{center}

            \item Cambiar el capacitor de 470 µF por un capacitor de 2,200 µF y realizar las mismas
            mediciones del inciso a), registrar los valores obtenidos en la Tabla 1.11; realizas las
            mediciones del inciso b) y c), registrar la medición en la Tabla 1.12 y graficar las señales en
            la Fig. 1.22 y Fig. 1.23; y realizar las mediciones del inciso d), registrar la medición en la
            Tabla 1.12 y graficar las señales en la Fig. 1.24.

            \begin{center}
                \includegraphics*[scale=0.45]{fig1-22.png}
                \includegraphics*[scale=0.45]{fig1-23.png}
                \includegraphics*[scale=0.45]{fig1-24.png}
                \includegraphics*[scale=0.2]{img1-24.jpg}
            \end{center}


        

        
        \end{enumerate}


        \end{enumerate}






















        \section*{Análisis matemáticos}
        \label{sec:calculos}

        \section*{Simulación de práctica}
        \label{sec:simulador}

        \section*{Análisis de resultados}
        \label{sec:resultados}
        Realizar los cálculos de los circuitos de las secciones 3.1, 3.3 y 3.5 para obtener los valores de $V_0$,
        $I_0$ y $V_{PI}$, que corresponden a los rectificadores.\par
        \begin{itemize}
            \item Rectificador de media onda
            \item Rectificador de onda completa con derivación central
            \item Rectificador de onda completa tipo puente
        \end{itemize}


        Y realizar los cálculos de los circuitos de las secciones 3.2, 3.4 y 3.6 para obtener los valores de $V_0$,
        $I_0$, $V_{max}$, $V_{min}$ y $\Delta V_0$, que corresponden a los rectificadores.
        \begin{itemize}
            \item Rectificador de media onda con filtro
            \item Rectificador de onda completa con derivación central con filtro
            \item Rectificador de onda completa tipo puente con filtro
        \end{itemize}

        \section*{Cuestionario}
        \label{sec:cuestionario}
        \begin{enumerate}
            \item \textbf{Menciona la importancia de los rectificadores de voltaje.}\par
                Los rectificadores de voltaje son de gran importancia en la electrónica aplicada ya que son la única forma de convertir
                la corriente eléctrica de alterna, que es la forma en la que se distribuye en las redes mundiales, a directa, que es como 
                se utiliza en la mayoría de los circuitos eléctricos. Por lo que sin estos, toda la infraestructura eléctrica en el mundo quedaría inservible. 
            \item \textbf{Explica la diferencia que existe entre un rectificador de media onda y uno de onda completa}\par
                El rectificador de media onda utiliza un solo diodo para rectificar la señal, mientras que el de onda completa utiliza dos, por el mismo motivo,
                el rectificador de media onda `desperdicia' la mitad negativa de la señal dadas sus limitaciones, mientras que, a pesar de utilizar la caída de tensión de
                dos diodos, el rectificador de onda completa aprovecha mejor la mitad negativa de la señal de entrada. También es importante mencionar que el rectificador de onda completa necesita de 
                un transformador con derivación central para funcionar.
            \item \textbf{¿Cuál es la diferencia de un rectificador de onda complete con derivación central y del tipo puente?}\par
                El rectificador tipo puente no requiere de un transformador con derivación central, no obstante requiere de un arreglo con dos diodos más que el rectificador con derivación
                central. El arreglo de diodos del rectificador tipo puente permite a la corriente fluir en ambas mitades del ciclo alternante sin necesidad de dividir el voltaje, por lo que es significativamente
                más eficiente que su contraparte y por las mismas razones más ampliamente usado en la electrónica.
            \item \textbf{¿Cómo se mide el voltaje de salida del rectificador?}\par
                Se colocan las puntas del multímetro en posición para medir voltaje en la resistencia de carga del circuito; es decir, en paralelo a sus terminales y en configuración
                de corriente directa, dado que la señal ya se encuentra rectificada.
            \item \textbf{¿Cómo se mide el voltaje de rizo del rectificador?}\par
                Necesita utilizarse el osciloscopio, se coloca en paralelo a las terminales de la resistencia de carga en modo de medición de corriente alterna.
            \item \textbf{Establecer las ventajas que tienen los rectificadores con filtro sobre los rectificadores con filtro sobre los rectificadores sin filtro.}\par    
                Un rectificador con filtro tiene la característica de que la onda rectificada, es decir, aquella medida en la resistencia de carga del circuito tiene una pulsación significativamente menos 
                pronunciada que un rectificador sin filtro; en otras palabras, el efecto del filtro es que el voltaje de salida es bastante más parecido a un voltaje constante que a una serie de pulsaciones, que 
                es como se observa la onda en el rectificador sin filtro, mejorando significativamente el $V_{rms}$ de salida.
        \end{enumerate}

        \section*{Conclusiones}
        \label{sec:conclusiones}
        \textbf{Carpio Becerra Erick Gustavo}\par
        \textbf{Espinoza Vera Fransisco}\par
        \textbf{Portela Nájera Jesús Bambino}\par
        El proceso de rectificación de señales de corriente alterna es indispensable para la infraestructura eléctrica en todo el mundo
        , y por esta razón es que esta es una de las principales aplicaciones de los diodos, ya que aprovechan a la perfección las características de su funcionamiento 
       mediante la unión de un material P y un material N. Por otro lado, la implementación del capacitor es una manera simple, eficiente y barata de mejorar la calidad
       de la señal en corriente directa en la salida de los rectificadores, por lo que no encuentro escenario en el que sea una alternativa viable implementar un rectificador 
       sin filtro.

        \section*{Referencias}
        \label{sec:referencias}
        \begin{itemize}
            \item Electricalacademia.com, 2018. [Online]. Available:\par\ https://electricalacademia.com/electronics/diode-definition-working-principle-construction/. [Accessed: Apr. 9, 2023]
            \item Matsusada Precision Inc, How diodes work and what used for. Matsusada Precision, Nov. 10, 2021. [Online]. Available: https://www.matsusada.com/column/\par\ words\_diode.html. [Accessed: Apr. 09, 2023]
            \item T. R. Kuphaldt, ``Rectifier Circuits´´ Allaboutcircuits.com, Feb. 12, 2015. \par\ [Online]. Available: https://www.allaboutcircuits.com/textbook/\par\ semiconductors/chpt-3/rectifier-circuits/. [Accessed: Apr. 11, 2023]
        \end{itemize}
        
\end{document}